\documentclass{article}
\usepackage[utf8]{inputenc}
\usepackage[mmddyyyy]{datetime}

\title{Homework 4}
\author{Leon Kloker}
\date{\today}

\begin{document}

\section{Class Description}

The Truss class can be used to calculate beam forces in a statically
determined truss. The constructor takes two directories as input, one
for the joints data file and one for the beams data file, which have
to be formatted similarly as the example files. The constructor
reads the data from both files using the \texttt{read\_beams} and 
\texttt{read\_joints} methods and saves them in a \texttt{\_joints} 
and \texttt{\_beams} dictionary, respectively. \\
Then, the \texttt{statical\_determinancy} method checks if the method 
of joints can be used to calculate the beam forces in the truss and 
raises a RuntimeError if this is not the case.\\ 
Lastly, the \texttt{calculate\_forces} method is invoked,
which creates a sparse csr matrix in order to save the coefficients of
the beam forces and reaction forces that appear when considering the static
equilibrium at each joint. When the resulting matrix is singular, the method
raises a RuntimeError as the truss is possibly unstable. If this is not the 
case, the linear equation system is solved with a sparse solver and the 
beam and reaction forces are saved in a \texttt{\_forces} array. When the print
function is called on a Truss object, the calculated beam
forces are listed.

\section{Exemplary Usage}

The main.py file can be used print the beam forces of a truss in the 
following fashion:\\
\texttt{\$ python3 main.py cme211-hw4-files/truss1/joints.dat\\ 
cme211-hw4-files/truss1/beams.dat\\
~Beam~~~~Force\\
-----------------\\
~~~~1~~~~~~~0.000\\
~~~~2~~~~~~-1.000\\
~~~~3~~~~~~~0.000\\
~~~~4~~~~~~-1.000\\
~~~~5~~~~~~~0.000\\
~~~~6~~~~~~~0.000\\
~~~~7~~~~~~-0.000\\
~~~~8~~~~~~-1.414}
\vspace{0.5cm}\\
If an optional third directory is specified, the truss geometry 
is plotted and saved in this directory.\\
Time spent 6 hours.
\end{document}
